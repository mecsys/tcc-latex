% $Id: resumo.tex,v 1.1 2003/04/10 23:12:59 gweber Exp $
%%%%%%%%%%%%%%%%%%%%%%%%%%%%%%%%%%%%%
%% Resumo	
%% Copyright 2003 Dehon Charles Regis Nogueira.
%% Este documento � distribu�do nos termos da licen�a
%% descrita no arquivo LICENCA que o acompanha.
%%%%%%%%%%%%%%%%%%%%%%%%%%%%%%%%%%%%%

\begin{resumo}

% Apresenta��o concisa dos pontos relevantes, dando uma visao rapida e
% clara do conte�do do trabalho.

Neste trabalho foram realizadas medidas de constante dielétrica e perda, através da técnica de espectroscopia de impedância, e do parâmetro de transmissão, pela técnica de espectroscopia de microondas, de várias amostras de gasolina C -- que possui álcool etílico em sua composição, ao contrário da gasolina A que é isenta de álcool -- e de outros líquidos, utilizando uma célula de combustível. Foram coletadas 3 amostras de gasolina C de 5 postos de combustível situados no município de Fortaleza. Entre cada uma das três coletas, houve um período aproximado de 2 meses. A partir dos dados obtidos pelos dois métodos citados, foram elaboradas curvas de atenuação. Além das amostras dos postos locais, também foram estudadas algumas outras provenientes do Laboratório de Combustíveis e Lubrificantes da UFC, o LCL. Cada uma dessas amostras possui um laudo, o qual descreve suas especificaçães segundo diversos parâmetros normatizados, cujos resultados foram comparados com o que foi observado através das técnicas de medidas elôtricas. Realizou-se, ainda, o estudo de 3 misturas de líquidos: gasolina C com álcool etílico hidratado, gasolina C com solvente aguarrás e álcool etílico com água destilada. O comportamento das propriedades elétricas dessas misturas foi analisado e comparado com modelos teôricos existentes na literatura. Além da gasolina C, os seguintes derivados de petróleo também tiveram suas propriedades elétricas analisadas: uma amostra de óleo diesel, duas amostras de marcas diferentes de querosene, uma de solvente comercial, uma de gasolina tipo A e uma amostra de aguarrás. %Foi comparado, ainda, o valor do par�metro de transmiss�o de uma amostra de gasolina medido com a c�lula de combust�veis da UFC com o valor obtido com outra amostra medido com a utiliza��o da c�lula de combust�veis da UFPe.

\end{resumo}
