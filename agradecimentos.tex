% $Id: agradecimentos.tex,v 1.1 2003/04/10 23:12:59 gweber Exp $

%%%%%%%%%%%%%%%%%%%%%%%%%%%%%%%%%%%%%
%% Agradecimentos
%% Copyright 2003 Dehon Charles Regis Nogueira.
%% Este documento � distribu�do nos termos da licen�a
%% descrita no arquivo LICENCA que o acompanha.
%%%%%%%%%%%%%%%%%%%%%%%%%%%%%%%%%%%%%

\chapter*{Agradecimentos}


% Agradecimentos - � so para as pessoas que contribuiram relevantemente
% para a elabora��o do trabalho

Dedico meus sinceros agradecimentos para:

-- o professor doutor Antônio Sérgio Bezerra Sombra, pela orienatação e incentivo;

-- a equipe do Laboratôrio de Óptica Não-Linear e Ciência de Materiais do Departamento de Física da UFC, em especial aos colega doutorandos Rinaldo e Silva de Oliveira e Ana Fabíola Leite Almeida, pela ajuda em diversos momentos;

-- o coordenador do curso de Mestrado em Engenharia e Ciência de Materiais, professor doutor Lindberg Lima Gonçalves, pelo apoio sempre manifestado;

-- o professor doutor Hamilton Ferreira Gomes de Abreu, pela oportunidade de juntar-me à equipe da ANP e à professora doutora Mônica Cavalcante Sá de Abreu pela ajuda, entre outras coisas, na obtenção de amostras;

-- o professor doutor Hosiberto Batista de Sant'Ana, pelo auxílio nos tópicos referentes à Engenharia de Petróleo;

-- a minha esposa Júlia, pela revisão deste trabalho;

-- a Agência Nacional de Petróleo, pela oportunidade de realização deste trabalho;

-- o Departamento de Eletrônica e Sistemas da UFPe, nas pessoas do professor doutor Edval J. P. Santos e do mestrando Victor Miranda da Silva, pela recepção e auxílio durante minha estada em Recife;

-- o Laboratôrio de Lubrificantes e Combustíveis da UFC, nas pessoas de Stílio Menezes Rola Jr. e Sandra Lidiane Mota da Silva, pelas amostras fornecidas e pelo auxílio em diversos momentos de dúvida;

-- a Lubnor, na pessoa do engenheiro Paulo de Almeida Luz, pelas amostras de combustível fornecidas;

-- todos os colegas do Mestrado em Engenharia e Ciência de Materiais da UFC.
