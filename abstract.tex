% $Id: abstract.tex,v 1.1 2003/04/10 23:12:59 gweber Exp $

%%%%%%%%%%%%%%%%%%%%%%%%%%%%%%%%%%%%%
%% Abstract
%% Copyright 2003 Dehon Charles Regis Nogueira.
%% Este documento � distribu�do nos termos da licen�a
%% descrita no arquivo LICENCA que o acompanha.
%%%%%%%%%%%%%%%%%%%%%%%%%%%%%%%%%%%%%


\begin{abstract}

% Apresenta��o concisa dos pontos relevantes, dando uma visao rapida e
% clara do conte�do do trabalho.

This dissertation describes the measurements of the dielectric constant and loss by impedance spectroscopy technique and the transmission parameter by the microwave spectroscopy technique using some samples of C gasoline, which has ethanol in its composition, and other liquids using a fuel cell. Three samples of C gasoline were collected in 5 gas stations placed in Fortaleza City. There was a period of two months between each collection. Attenuation curves were set with the information obtained by the two methods described before.  Samples obtained by the UFC Fuels and Lubricants Laboratory, LCL, were also studied. Each of these samples had a certificate describing its specifications according to many normative parameters, which results were compared to the information observed using the electrical measurement techniques. There was also the analysis of three liquid mixtures: C gasoline with hydrated ethylic alcohol, C gasoline with \textit{aguarrás} solvent and ethylic alcohol with distilled water. The behavior of the electrical properties of these mixtures was analyzed and compared to theoretical existent models in the literature. Other petroleum derivatives had their electrical properties analyzed: a sample of diesel, two different commercial samples of kerosene, one of commercial solvent, a sample of A gasoline, that has no alcohol in its composition, and an \textit{aguarrás} sample. %It was also compared the transmission parameter of a gasoline sample measured with the UFC fuel cell and the data obtained with the UFPe fuel cell.


%This dissertation describes measurements of the dielectric constant and loss -- done by impedance spectroscopy -- and the transmission parameter -- done by microwave spectroscopy -- for some samples of type C gasoline -- that has ethanol in its composition -- and other liquids using a fuel cell. Three samples of type C gasoline were collected in 5 gas stations placed in Fortaleza City. Between each collect there were a period of two months. With the information obtained by the two methods described before, attenuation curves were plotted.  Samples obtained by the UFC Fuels and Lubricants Laboratory, LCL, were also studied. Each of these samples had a certificate describing its specifications according with many normalized parameters, where results were compared with the informations observed using the electrical measurement techniques. There where also the analysis of three liquid mixtures: type C gasoline with hydrated ethylic alcohol, type C gasoline with \textit{aguarr�s} solvent and ethylic alcohol with distilled water. The behavior of the electrical properties of these mixtures was analyzed and compared with theoretical models existent in the literature. Others petroleum derivatives had their electrical properties analyzed: a sample of diesel, two different commercial samples of kerosene, one of commercial solvent, a sample of type A gasoline -- that has no alcohol in its composition -- and a \textit{aguarr�s} sample. %It was also compared the transmission parameter of a gasoline sample measured with the UFC fuel cell and the data obtained with the UFPe fuel cell.

\end{abstract}
